\documentclass{article}

\usepackage[utf8]{inputenc}
\usepackage[spanish]{babel}
\usepackage{hyperref}
\usepackage{listings}
\lstset{
    inputencoding = utf8,  % Input encoding
    extendedchars = true,  % Extended ASCII
    literate      =        % Support additional characters
      {á}{{\'a}}1  {é}{{\'e}}1  {í}{{\'i}}1 {ó}{{\'o}}1  {ú}{{\'u}}1
      {Á}{{\'A}}1  {É}{{\'E}}1  {Í}{{\'I}}1 {Ó}{{\'O}}1  {Ú}{{\'U}}1
      {à}{{\`a}}1  {è}{{\`e}}1  {ì}{{\`i}}1 {ò}{{\`o}}1  {ù}{{\`u}}1
      {À}{{\`A}}1  {È}{{\`E}}1  {Ì}{{\`I}}1 {Ò}{{\`O}}1  {Ù}{{\`U}}1
      {ä}{{\"a}}1  {ë}{{\"e}}1  {ï}{{\"i}}1 {ö}{{\"o}}1  {ü}{{\"u}}1
      {Ä}{{\"A}}1  {Ë}{{\"E}}1  {Ï}{{\"I}}1 {Ö}{{\"O}}1  {Ü}{{\"U}}1
      {â}{{\^a}}1  {ê}{{\^e}}1  {î}{{\^i}}1 {ô}{{\^o}}1  {û}{{\^u}}1
      {Â}{{\^A}}1  {Ê}{{\^E}}1  {Î}{{\^I}}1 {Ô}{{\^O}}1  {Û}{{\^U}}1
      {œ}{{\oe}}1  {Œ}{{\OE}}1  {æ}{{\ae}}1 {Æ}{{\AE}}1  {ß}{{\ss}}1
      {ẞ}{{\SS}}1  {ç}{{\c{c}}}1 {Ç}{{\c{C}}}1 {ø}{{\o}}1  {Ø}{{\O}}1
      {å}{{\aa}}1  {Å}{{\AA}}1  {ã}{{\~a}}1  {õ}{{\~o}}1 {Ã}{{\~A}}1
      {Õ}{{\~O}}1  {ñ}{{\~n}}1  {Ñ}{{\~N}}1  {¿}{{?`}}1  {¡}{{!`}}1
      {°}{{\textdegree}}1 {º}{{\textordmasculine}}1 {ª}{{\textordfeminine}}1
      {£}{{\pounds}}1  {©}{{\copyright}}1  {®}{{\textregistered}}1
      {«}{{\guillemotleft}}1  {»}{{\guillemotright}}1  {Ð}{{\DH}}1  {ð}{{\dh}}1
      {Ý}{{\'Y}}1    {ý}{{\'y}}1    {Þ}{{\TH}}1    {þ}{{\th}}1    {Ă}{{\u{A}}}1
      {ă}{{\u{a}}}1  {Ą}{{\k{A}}}1  {ą}{{\k{a}}}1  {Ć}{{\'C}}1    {ć}{{\'c}}1
      {Č}{{\v{C}}}1  {č}{{\v{c}}}1  {Ď}{{\v{D}}}1  {ď}{{\v{d}}}1  {Đ}{{\DJ}}1
      {đ}{{\dj}}1    {Ė}{{\.{E}}}1  {ė}{{\.{e}}}1  {Ę}{{\k{E}}}1  {ę}{{\k{e}}}1
      {Ě}{{\v{E}}}1  {ě}{{\v{e}}}1  {Ğ}{{\u{G}}}1  {ğ}{{\u{g}}}1  {Ĩ}{{\~I}}1
      {ĩ}{{\~\i}}1   {Į}{{\k{I}}}1  {į}{{\k{i}}}1  {İ}{{\.{I}}}1  {ı}{{\i}}1
      {Ĺ}{{\'L}}1    {ĺ}{{\'l}}1    {Ľ}{{\v{L}}}1  {ľ}{{\v{l}}}1  {Ł}{{\L{}}}1
      {ł}{{\l{}}}1   {Ń}{{\'N}}1    {ń}{{\'n}}1    {Ň}{{\v{N}}}1  {ň}{{\v{n}}}1
      {Ő}{{\H{O}}}1  {ő}{{\H{o}}}1  {Ŕ}{{\'{R}}}1  {ŕ}{{\'{r}}}1  {Ř}{{\v{R}}}1
      {ř}{{\v{r}}}1  {Ś}{{\'S}}1    {ś}{{\'s}}1    {Ş}{{\c{S}}}1  {ş}{{\c{s}}}1
      {Š}{{\v{S}}}1  {š}{{\v{s}}}1  {Ť}{{\v{T}}}1  {ť}{{\v{t}}}1  {Ũ}{{\~U}}1
      {ũ}{{\~u}}1    {Ū}{{\={U}}}1  {ū}{{\={u}}}1  {Ů}{{\r{U}}}1  {ů}{{\r{u}}}1
      {Ű}{{\H{U}}}1  {ű}{{\H{u}}}1  {Ų}{{\k{U}}}1  {ų}{{\k{u}}}1  {Ź}{{\'Z}}1
      {ź}{{\'z}}1    {Ż}{{\.Z}}1    {ż}{{\.z}}1    {Ž}{{\v{Z}}}1
      % ¿ and ¡ are not correctly displayed if inconsolata font is used
      % together with the lstlisting environment. Consider typing code in
      % external files and using \lstinputlisting to display them instead.      
  }

\title{Preprocesamiento, tokenización y extracción de información}
\author{Fabián Villena}
\date{Junio 2025}

\begin{document}

\maketitle

El análisis y minería de texto son habilidades fundamentales en el ámbito de la ciencia de datos. En esta tarea pondrá en práctica técnicas de preprocesamiento, tokenización, filtrado y extracción de información estructurada desde documentos en lenguaje natural.

Para este ejercicio, trabajará con un conjunto de archivos de texto que contienen descripciones clínicas reales, pero las técnicas que aplicará son extensibles a cualquier dominio con textos libres.

El equipo de tecnologías de la organización ha exportado cientos de archivos de texto plano (\texttt{.txt}), cada uno correspondiente a una hipótesis diagnóstica. Un ejemplo de contenido de archivo es el siguiente:

\begin{lstlisting}[breaklines=true, extendedchars=true,numbers=left,frame=single]
- TRASTORNO DE LA REFRACCIÓN, NO ESPECIFICADO

 paciente de 71 años, con antecedentes de hta en tto, diabetes insulinodependiente, dislipidemia, hipotiroidismo en tto, enfermedad renal cronica etapa iii,tabaquismo cronico importante, en febrero de este año lo suspendio. Refiere que tiene principios de Alzheimer y parkinson?????? NO SALE REGISTRO DE DIAGNOSTICOS. Refiere que necesita ic a oftalmologo. Tiene astigmatismo y miopia, ocupa lentes para ambos trastornos de viciorefraccion, refiere que hace 4 meses que ve borroso utilizando lentes ópticos.  Fue operada hace mas de 2 años por retinopatia diabetica en ambos ojos.

Al ex fisico: No observo ojo rojo. pupilas isocoricas y reactivas. no observo opacidades corneales. RFM presente. agudeza visual conservada.
\end{lstlisting}

El conjunto completo de textos está publicado aquí:

\begin{center}
  \url{https://doi.org/10.5281/zenodo.7555181}
\end{center}


\vspace{1em}
\section*{Tarea Práctica}

Siga los pasos y responda en un \emph{Jupyter Notebook} usando Python y las bibliotecas que estime convenientes. Recuerde entregar un Notebook ya ejecutado que muestre los resultados, dado que su Notebook no va a ser ejecutado.

\begin{enumerate}

  \item \textbf{Carga de los textos}
    \begin{enumerate}
        \item Importe todos los archivos de texto (\texttt{.txt}) y cuente cuántos hay en total (\textbf{N} documentos).
        \item Para cada documento, registre la cantidad de caracteres. (1 punto)
        \item Asegúrese de que los caracteres especiales y con tildes se importen correctamente (codificación).
    \end{enumerate}

  \item \textbf{Preprocesamiento y análisis léxico}
    \begin{enumerate}
        \item Implemente y explique una función de preprocesamiento, por ejemplo: pasar a minúsculas, eliminar puntuación, normalización de espacios, etc. (\texttt{str} $\rightarrow$ \texttt{str}). (1 punto)
        \item Implemente una función de tokenización (\texttt{str} $\rightarrow$ \texttt{List[str]}) para dividir el texto en palabras. (1 punto)
        \item Elabore una lista de \emph{stopwords} (palabras vacías) en español, y filtre las palabras del texto eliminando estas. (1 punto)
        \item Visualice las palabras más frecuentes en el corpus, idealmente con una \textit{nube de palabras} (\emph{wordcloud}). (1 punto)
    \end{enumerate}

  \item \textbf{Extracción de información mediante patrones (expresiones regulares)}
    \begin{enumerate}
        \item Usando expresiones regulares, detecte menciones de las siguientes enfermedades en los textos. Para cada una, diseñe un patrón que contemple formas alternativas o sinónimos. Se proveen los códigos CIE-10 y vínculos a Wikidata para su referencia (1 puntos):
          \begin{enumerate}
            \item[\href{https://www.wikidata.org/wiki/Q41861}{I10}] Hipertensión esencial (primaria)
            \item[\href{https://www.wikidata.org/wiki/Q3025883}{E11}] Diabetes mellitus tipo 2
            \item[\href{https://www.wikidata.org/wiki/Q128581}{C50}] Neoplasia maligna de mama
            \item[\href{https://www.wikidata.org/wiki/Q736715}{N18}] Enfermedad renal crónica
            \item[\href{https://www.wikidata.org/wiki/Q11081}{G30}] Enfermedad de Alzheimer
          \end{enumerate}
        \item Para cada entidad, indique cuántos documentos la mencionan y calcule en todos los casos la prevalencia (documentos que la mencionan/N).
    \end{enumerate}

\end{enumerate}

\end{document}